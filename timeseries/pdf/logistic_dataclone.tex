\documentclass{article}
\usepackage{graphicx}
\usepackage{amsmath}
\usepackage{mathrsfs}
\usepackage[margin=1in]{geometry} % Reduce margins to use more of the page

\begin{document}
\begin{titlepage}
   \vspace*{\stretch{1.0}}
   \begin{center}
      \Large\textbf{Deciduousness Timing Analysis}\\
      \large\textit{Vicente Vasquez}
   \end{center}
   \vspace*{\stretch{2.0}}
\end{titlepage}

\section{Simulation of Data}

we have collected 7 years of drone data.  Each year we have flown the drone 12 times at 30 day intervals.
We have mapped the tree crowns of 10 known individuals of the same species. 
Each individual tree crown was observed in all flight dates. 
All tree crowns had their leaf coverage percentage estimated utilizing a machine learning algorithm on the drone imagery.
The leaf coverage percentage values range from 0 to 1, where 0 means no leaves and 1 means full leaf coverage.

\begin{verbatim}
    n.years  <- 7
    one.year <- seq(from=1,to=365,by=30)
    samp.days <- rep(one.year,n.years)
    n.inds <- 10
    all.days <- rep(samp.days,n.inds)
    n <- length(all.days)
    year.id  <- rep(rep(1:n.years, each = length(one.year)), n.inds)
    indv.id  <- rep(1:n.inds, each = length(samp.days))
\end{verbatim}

Once we have simulated the sampling design, we can simulate the data.
We will use a simple logistic function to simulate the leaf drop event.

\begin{equation}
\begin{split}
    leafDrop(kd, Td, x) = \frac{1}{1 + e^{kd(x - Td)}}\\
    \mu = kd(x - Td)
\end{split}
\end{equation}


\begin{verbatim}
    logit.pf <- function(kd,Td,x){
        out <- kd*(x-Td)
        return(out)
}
\end{verbatim}

Where $kd$ is the rate of leaf drop, $Td$ is the day of year when 50\% of leaves are dropped, and $x$ is the day of year.

We will simulate the following parameters:
\begin{verbatim}
    sigsq <- 0.45  #noise levels
    kd <- 0.1      #leaf drop rate
    Td <- 100      #day of year when 50% of leaves are dropped
    mu.true <- logit.pf(kd=kd,Td=Td,x=all.days)  
\end{verbatim}

Sample from a normal distribution with mean $\mu$ and standard deviation $\sqrt{sigsq}$ to get the observed leaf coverage percentage values.
\begin{verbatim}
    norm.samps <- rnorm(n=n, mean=mu.true, sd=sqrt(sigsq))
    y.sims <- 1/(1+exp(norm.samps))
\end{verbatim}

Perform a logit transformation to map 0-1 values to the real line.
\begin{verbatim}
    test.data <- log(1-y.sims) - log(y.sims)
\end{verbatim}

Clone data along a new dimension for use in the model.
\begin{verbatim}
    data4dclone <- list(K=1, X=dcdim(data.matrix(test.data)), n=n, days=all.days)
\end{verbatim}

Define model formula and parameters.

\begin{verbatim}
    cl.seq <- c(1,4,8,16);
    n.iter<-10000;n.adapt<-5000;n.update<-100;thin<-10;n.chains<-3;

    out.parms <- c("kd", "Td", "sigsq")
    leaves.dclone <- dc.fit(data4dclone, params=out.parms, model=leaves, n.clones=cl.seq,
                            multiply="K",unchanged="n",
                            n.chains = n.chains, 
                            n.adapt=n.adapt, 
                            n.update=n.update,
                            n.iter = n.iter, 
                            thin=thin,
                            inits=list(lkd=log(0.2), ltd=log(40))
                            )
\end{verbatim}

\begin{figure}
    \centering
    \includegraphics[width=0.8\textwidth]{../../plots/simulated_7years.png}
    \caption{Simulated leaf coverage percentage data for 10 individuals over 7 years. Each color represents a different individual tree crown.}
\end{figure}

\section{Results}

Posterior summaries after 16 clones:
\begin{table}[h!]
\centering
\begin{tabular}
{|c|c|c|c|c|}
\hline
Parameter & True Value & Posterior Mean & Standard deviation & 95\% Credible Interval\\
\hline
$kd$ & 0.1 & 0.1002 & 5.194e-05 &(0.095, 0.109) \\
$Td$ & 100 & 100.3830 & 7.068e-02 &(100.2455, 100.5213) \\
$sigsq$ & 0.45 & 0.4699 & 5.738e-03 &(0.4699, 0.4928) \\
\hline
\end{tabular}
\end{table}


\begin{figure}
    \centering
    \includegraphics[width=0.8\textwidth]{../../plots/results_intercept.png}
    \caption{Posterior distributions of model parameters after 1, 4, 8, and 16 clones. The variance decrease at 1/k rate}
\end{figure}

\begin{figure}
    \centering
    \includegraphics[width=0.8\textwidth]{../../plots/results_intercept_log.png}
    \caption{Parameter estimates in the log scale. Convergence is shown as roughly decrease of 1/k in the posterior variance as the number of clones increase.}
\end{figure}


\subsection{Real data analysis}

Cavallinesia planatifolia is a obligate deciduous tree species found in the Barro colorado island in Panama. 
We observed 16 individual tree crowns of this species between 04-04-2018 and 03-18-2024 using drone imagery.
The leaf coverage percentage was estimated using a machine learning algorithm on the drone imagery.
Here we have subseted all observations from the leaf drop event. 
Here we have shifted the day of year such as that DOY 1 is at september 1st of each year.
The inflection point (Td) is expected to occur close to January 1, thus it was neccessary to account for circularity of the data.
\begin{figure}
    \centering
    \includegraphics[width=0.8\textwidth]{../../plots/cava_9shifted.png}
    \caption{Observed leaf coverage percentage data for 16 individuals of Cavallinesia planatifolia over 7 years. Each color represents a different individual tree crown.}
\end{figure}


Posterior summaries after 16 clones:
\begin{table}[h!]
\centering
\begin{tabular}
{|c|c|c|c|}
\hline
Parameter & Posterior Mean & Standard deviation & 95\% Credible Interval\\
\hline
$kd$ & 0.05626 & 0.0004888 & (0.05532, 0.05722) \\
$Td$ & 125.52889 & 0.6768785 & (124.17523, 126.85004) \\
$sigsq$ & 12.19795 & 0.1889067 & (11.84159, 12.58489) \\
\hline
\end{tabular}
\end{table}


\begin{figure}
    \centering
    \includegraphics[width=0.8\textwidth]{../../plots/cava_estimate.png}
    \caption{Posterior distributions of model parameters after 1, 4, 8, and 16 clones. The variance decrease at 1/k rate}
\end{figure}
\begin{figure}
    \centering
    \includegraphics[width=0.8\textwidth]{../../plots/cava_estimate_log.png}
    \caption{Parameter estimates in the log scale. Convergence is shown as roughly decrease of
    1/k in the posterior variance as the number of clones increase.}
\end{figure}

By defining the following events: Start of leaf drop (SOD) as the day when 10\% of leaves are dropped, and End of leaf drop (EOD) as the day when 90\% of leaves are dropped, and Td as the day when 50\% of leaves are dropped, we can calculate the following:
\itemize{
    \item SOD = November 26th 
    \item Td = January 4th
    \item EOD = February 12th
}
\begin{figure}
    \centering
    \includegraphics[width=0.8\textwidth]{../../plots/cava_9shifted_result.png}
    \caption{Predicted line using posterior mean estimates for kd and Td overlaid on observed data.}
\end{figure}


\end{document}