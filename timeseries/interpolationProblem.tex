\documentclass{article}
\usepackage{graphicx}
\usepackage{amsmath}
\usepackage{mathrsfs}
\usepackage[margin=1in]{geometry} % Reduce margins to use more of the page
\begin{document}

\subsection{Problem Statement}
I aim to model a seasonal leaf coverage curve for multiple trees across multiple phenological years.
Leaf coverage is a continuous variable ranging from 0 to 100, where 0 means no leaves and 100 means full leaf coverage.
Phenological years are defined as a full cycle of leaf coverage, which may not align with calendar years.
For example, some trees have bimodal yearly cycles, with two distinct periods of leaf coverage increase and decrease.
Thus, their phenological year may have a duration of half a calendar year.
In this study, individual trees are distinct exposed tree crowns that are observed from drone imagery.
The observations are made at partially regular intervals, and all trees are observed at the same time.
The full span of the observations ranges from  April 4, 2018 to March 18, 2024.
Observations are made at a monthly frequency between April 2018 and January 2023 and 
weekly frequency between February 2023 and March 2024.
An exception exists between February 27, 2020 and May 27, 2020 where no observations were made due to COVID-19 lockdowns.

From here onwards, I will refer to a "deciduousness event" as the period of time in which a tree drops its leaves , stays dormant and then flushes new leaves.
A leaf drop event is defined as the period of time in which a tree drops its leaves.
A leaf flush event is defined as the period of time in which a tree flushes new leaves.
A dormancy event is defined as the period of time in which a tree has no leaves. 
All of the events have a duration in days.
Leaf drop and leaf flush events have a duration and rate of change in leaf coverage.
Dormancy events have a duration but we expect no change in leaf coverage.
Fully leafed events have a duration but we expect very little change in leaf coverage attributed to leaf exchange during the phenological year.



The main questions are:
\begin{itemize}
    \item Does phenological year has an effect on the timing of leaf deciduousness events?
    \item What is the variability associated to individual trees in the timing and duration of leaf deciduousness events?
\end{itemize}

From the stated questions, we can identify two extreme cases.
\begin{itemize}
    \item A group of trees which shows no variation in the timing of leaf deciduousness events. In other words, all trees drop and flush leaves at the same time every year.
    \item A group of trees which shows cofounded intraspecific and interspecific variation in the timing of leaf deciduousness events. In other words, trees drop and flush leaves at different times every year.
\end{itemize}

The real challenge arises from the sparse observations which may not capture the transition of leaf drop and leaf flush events.
For example, if a tree is observed with full leaf coverage on day 1 and no leaves on day 30, we do not know when the tree dropped its leaves.

Let $I$ be the set of all individual trees observed.
Let $T$ be the set of all observation dates.
The phenological year for each observation date $t \in T$ is defined as $year(t)$.
The response variable is leaf coverage $y$, which is a continuous variable ranging from 0 to 100.
All trees in $I$ are observed at all dates in $T$ and each observation has an associated leaf coverage value $y$.
For the sake of simplicity, we know that all trees are synchronous and drop their leaves at the same time every year.

Thus, we can define the problem as finding a function $f$ such that:
\begin{equation}
    y_{t, i} = f(t, i, year(t)) + \epsilon
\end{equation}

Each observation $y_{t, i}$ is either the result a human annotation of the leaf coverage observed in drone imagery or 
the result of a machine learning model prediction of leaf coverage from drone imagery.
The human annotation is assumed to be the ground truth and the ML model error is low in the extremes of the leaf coverage
 range (0 and 100) and high in the middle of the range.

\begin{equation}
    \epsilon = \epsilon_{human} + \epsilon_{model}
\end{equation}

\begin{equation}
    \epsilon_{model} =
    \begin{cases}
        3, & \text{if } 0 \geq y \leq 20 \\
        3, & \text{if } 80 \leq y \leq 100 \\
        30, & \text{if } 20 < y < 80\\
    \end{cases}
\end{equation}

The $y_{t, i}$ observations are roughly 80\% human annotations and 20\% ML model predictions.

\subsection{Simulation Example 1. Synchronous Trees}
Here I simulate the values of $y_{t, i}$ for a set of 20 trees observed at the dates in $T$.

\begin{figure}
    \centering
    \includegraphics[width=\textwidth]{C:/Users/Vicente/repo/ForestLandscapes/plots/simulated_synchronous.png}
    \caption{Simulated leaf coverage observations for 20 trees observed at the dates in $T$. The black line represents the true leaf coverage curve. The points represent the observations $y_{t, i}$ for each tree $i$ at each date $t$.}
    \label{fig:interpolation_problem}
\end{figure}

We can read in the data in R. First obtain the day of year and year from the date.
Then normalize the observed leaf coverage to be between 0 and 1, avoiding exact 0 and 1 values for numerical stability.

\begin{verbatim}
    simulated<- read.csv("timeseries\\simulated_phenophase_data.csv")
    simulated$day<- yday(simulated$date)
    simulated$year<- year(simulated$date)
    simulated$y_norm <- pmin(pmax(simulated$observed_leafing / 100, 1e-4), 1 - 1e-4)
\end{verbatim}

Then we can model the data using brms with a non-linear model smoothed cyclical splines basis to campture the seasonal pattern.
Even if we have completely missed the transition of leaf drop and leaf flush events in most years.
The weekly data from 2023 and 2024 should help to constrain the model.

\begin{verbatim}
    library(brms)
    fit_smooth <- brm(
    y_norm ~ s(day,k=20, bs="cc"),
    data = simulated,
    family = Beta(),
    chains = 4,
    iter = 4000,
    warmup = 1000,
    cores = 4
    )
    >Smoothing Spline Hyperparameters:
                Estimate Est.Error l-95% CI u-95% CI Rhat Bulk_ESS Tail_ESS
    sds(sday_1)     2.09      0.37     1.50     2.94 1.00      980     1781
    Regression Coefficients:
            Estimate Est.Error l-95% CI u-95% CI Rhat Bulk_ESS Tail_ESS
    Intercept     3.34      0.03     3.28     3.39 1.00     7197     7634
    Further Distributional Parameters:
        Estimate Est.Error l-95% CI u-95% CI Rhat Bulk_ESS Tail_ESS
    phi    37.37      1.57    34.36    40.50 1.00     7242     7541
\end{verbatim}

By defining the start of leaf drop event as the first day in which the predicted leaf coverage is below 0.9.
We can then extract the predicted values and find the first day below 0.9.
Similarly, we can find the start of leaf flush event as the first day in which the predicted leaf coverage is above 0.9.

\begin{verbatim}
    newdata <- data.frame(day = 1:365)
    pred_draws <- posterior_epred(fit_smooth, newdata = newdata)
    pred_mean <- posterior_epred(fit_smooth, newdata = newdata, re_formula= NA)
    pred_mean <- colMeans(pred_mean)
    first_below_09 <- which(pred_mean < 0.9)[1]
    first_above_09 <- which(pred_mean > 0.9 & (1:365) > first_below_09)[1]
    ggplot(newdata, aes(x = day, y = pred_mean)) +
    geom_line(color = "blue", size = 1) +
    geom_point(data = simulated, aes(x = day, y = y_norm, color = as.factor(year)), alpha = 0.5) +
    geom_line(data = df_true, aes(x = time, y = leafing / 100), color = "red", size = 1, linetype = "dashed") +
    geom_vline(xintercept = first_below_09, color = "black", size= 1) +
    geom_vline(xintercept = first_above_09,  color = "black", size= 1) +
    labs(x = "Day of Year", y = "Predicted y_norm") +
    theme_minimal() +
    xlim(1, 60)
\end{verbatim}
We have nailed the true timing of leaf drop and leaf flush events despite the sparse observations.
Given that we have pooled all trees together and assumed they are synchronous.
\begin{figure}
    \centering
    \includegraphics[width=0.8\textwidth, height=0.4\textheight, keepaspectratio]{C:/Users/Vicente/repo/ForestLandscapes/plots/simulated_smooth.png}
    \caption{Predicted leaf coverage curve from the brms model. The black vertical lines represent the start of leaf drop and end of leaf flush events. The red dashed line represents the true leaf coverage curve. The blue line represents the predicted leaf coverage curve from the brms model. The points represent the observations $y_{t, i}$ for each tree $i$ at each date $t$.}
    \label{fig:interpolation_problem_fit}
\end{figure}

\subsection{Simulation Example 2. Asynchronous Year}
Here I simulate the values of $y_{t, i}$ for a set of 20 trees observed at the dates in $T$.
2021 is a year in which all trees drop and flush leaves at the same time but 10 days after the usual time in other years.

\begin{figure}
    \centering
    \includegraphics[width=\textwidth]{C:/Users/Vicente/repo/ForestLandscapes/plots/simulated_2021_shifted.png}
    \caption{Simulated leaf coverage observations for 20 trees observed at the dates in $T$. The solid lines represent the true leaf coverage for synchronous trees and the points represent the observations $y_{t, i}$ for each tree $i$ at each date $t$. Notice that 2021 true line is shifted 10 days forward in time.}
    \label{fig:interpolation_problem_asynchronous}
\end{figure}

\end{document}
