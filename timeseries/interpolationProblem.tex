\documentclass{article}
\usepackage{graphicx}
\usepackage{amsmath}
\usepackage{mathrsfs}
\usepackage[margin=1in]{geometry} % Reduce margins to use more of the page
\begin{document}

\subsection{Problem Statement}

We have observed 3 individual trees of the same species before, during and after the dry season.
The species of tree is known to drop its leaves during the dry season and we aim to interpolate between observations the daily values.

Given a set of individuals $S={A,B,C}$, in which tree A was observed at days 2 and 4 with leafing values of 100 and 0, tree B was observed at days 1 and 3 with values 100 and 0, and tree C was observed at days 2 and 3 with values 100 and 0.
The data can be represented in a table as follows

\begin{table}[h]
    \centering
    \begin{tabular}{|c|c|c|}
        \hline
        GlobalID & Day & Leafing \\
        \hline
        A & 2 & 100 \\
        A & 4 & 0 \\
        B & 1 & 100 \\
        B & 3 & 0 \\
        C & 2 & 100 \\
        C & 3 & 0 \\
        \hline
    \end{tabular}
\end{table}

Rate of decay can be calculated for each individual as:
\begin{equation}
    k= \frac{\Delta y}{\Delta t} = \frac{y_{final} - L_{initial}}{t_{final} - t_{initial}}
\end{equation}
\begin{align}
    k_A &= \frac{0 - 100}{4 - 2} = -50 \\
    k_B &= \frac{0 - 100}{3 - 1} = -50 \\
    k_C &= \frac{0 - 100}{3 - 2} = -100
\end{align}
It can be interpreted as: "Trees A and B lose 50\% of their leaves per day, while tree C loses 100\% of its leaves per day".
But we do not know which one is correct, and the observed data may be influenced by various factors.

\begin{figure}[h]
    \centering
    \includegraphics[width=0.8\textwidth]{../plots/leafing_over_time.png}
    \caption{Leafing Over Time for Each Tree}
\end{figure}

Given a set observations $(t_i, y_i, l_i)$. Where $t_i$ is the day, $y_i$ is the leafing value (0-100) and $l_i$ is the label indicating if the observation was made by a human annotator or predicted from a model.

we aim to define $f(t)$ as a function that can interpolate the leafing values over time.

Each observation is either obtained from human annotators or is predicted from a Stochastic gradient boosting trees model. 
Values from humans annotators are considered to be ground truthed and have an associated error of zero.
Values predicted from the SGBT model have an associated error that can be defined as a function of the leafing value.

Let $\epsilon_i$ be the error associated with observation $O_i$. We can define the error as a function of the leafing value:

\begin{equation}
    f_{\text{error}}(y_i, l_i) = 
    \begin{cases}
        0 & \text{if } l_i = \text{annotator was a human} \\
        3 & \text{if } 0 \leq y \leq 20 \text{ or } 80 \leq y \leq 100 \\
        30 & \text{if } 20 < y < 80
    \end{cases}
\end{equation}

We can then fit a model to predict the leafing values given t and their associated errors.
In this case we will use a GAMM (Generalized Additive Mixed Model).

\end{document}